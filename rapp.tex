\documentclass[a4paper]{article}
\usepackage{style}
\usepackage{mcode}

\setlength{\droptitle}{15em}
\title{Laboration i Komponentfysik\\ Optoelektronik}

\author{Alexander Najafi \\ Linus Hellman}

\date{2014-04-13}

\begin{document}

\maketitle
\thispagestyle{empty}
\newpage

\tableofcontents
\newpage

\section{Inledning och bakgrund}

Denna lab är utformad för att ge en större förståelse för halvledarkomponenter och dess samverkan med ljus, både emission (ljusluminicens) samt absorption. Dioders kapacitet att emittera ljus kommer undersökas, samt om dennas emitterade ljus våglängd kommer att ha något samband med vilka våglängder som dioden kan absorbera om denna skulle användas som en detektor. Detta har ett extremt stort användningsområde tillexempel i kameror, där ljus ska översättas till digital data. Diodens uppbyggnad kommer även studeras då en egen GaP diod kommer tillverkas och analyseras. Denna kommer konstrueras med hjälp av en platta av GaP som kommer legeras med dels tenn dels zinklegerat guld. Detta kommer att dopa GaP-biten tillräckligt för att få en diods egenskaper. Vidare kommer en analys göras på en solcell för att undersöka vid vilken belastningsresistans denna uppnår sin maximal effekt. Belysning av en diod leder 


\end{document}
